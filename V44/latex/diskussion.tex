\section{Diskussion}
\label{sec:Diskussion}

% fehlerquellen:#

Bei der Betrachtung der Justage Messung fällt auf dass eine geringere Schrittweite
von Vorteil gewesen wäre für genauere Bestimmung der Strahlbreite z.B.
Auch die Bestimmung des Geometriewinkels aus dem Rocking Scan war aufgrund dieses Grundes
nicht sehr genau möglich. 
Der Geometriewinkel weicht um ca. $40 \%$ dem Theorie Wert (über den Z-scan) ab.

Der manuelle Fit war schwierig passend einzustellen so dass immer eine Seite
nicht richtig gepasst hat.
Demnach passt der Fit aber im Bereich von ca. $\SI{0.5}{\degree}$ bis $\SI{0.9}{\degree}$
optisch gut.
Aufgrund der manuellen Anpassung von 7 Parameter ist zu erwarten
keine sehr gute Anpassung hinzubekommen.
Ein weiterer Grund für Abweichungen von der Theorie Kurve sind Abnutzungen der Probe.

Für den kritischen Winkel lässt sich eine 
dementsprechend große Abweichungen von über 
$88 \%$ beobachten.  
$\alpha_\text{c,PS,gemessen} = \SI{0.081}{\degree}$ zu dem Literaturwert 
$\alpha_\text{c,PS,Literatur} = \SI{0.153}{\degree}$\cite{anleitung} 

Die Schichtdicke des Polysterol-Films, welche über den Parratt-Algorithmus bestimmt wurde,
 weicht jedoch nur um $5,19 \%$ zu der Schichdickenbestimmung über die Mittelung der Oszillationsminima ab.
Die Korrektur des Brechungsindizes von Silizium trifft annähernd den Literaturwert, $\delta_\text{PS,Literatur} = \num{7.6e-6}$\cite{anleitung},
 und weicht um $\approx 2,7 \%$ ab.

% d_\text{PS} = \SI{8.52+-0.26e-08}{\meter}
% z_2 &= \SI{8.10e-8}{\meter}\\

% \begin{align*}
%     \alpha_\text{c,PS} &= \SI{0.081}{\degree} \\
%     \alpha_\text{c,Si} &= \SI{0.220}{\degree}
% \end{align*}
% berechnet werden

