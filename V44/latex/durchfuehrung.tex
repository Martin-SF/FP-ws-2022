\section{Durchführung}
\label{sec:Durchführung}

Vor der eigentlichen Messung muss der Aufbau zunächst justiert werden. 

\subsection{Justage}

Die Justage des Aufbaus erfolgt in sechs Schritten.
Als erstes wird ein Detektorscan durchgeführt, wofür die Probe nicht im Strahlgang liegt. Der Detektor und die Röntgenröhre werden auf die
Position $\SI{0}{\degree}$ gefahren. Deren Lage wird nun variiert, um die tatsächliche Nulllage des Detektors zu finden. Dies wird getan, bis die Intensität des 
Primärstrahls das Maximum durchläuft. Genau diese Position ist die normierte Nullposition. Darauf folgt der erste Z-Scan.
Hier wird die Probenjustage angepasst. Die Probe wird in den Strahlengang gebracht und die Intensität $I$ wird für verschiedene
z-Positionen gemessen. Die Position der Probe wird variiert, bis $I$ auf den Wert $I = \frac{1}{2}I_\text{max}$ sinkt. Nun misst 
man den ersten Rockingscan. Der Detektor und die Röntgenröhre werden um die Probe bewegt. Es wird ein konstanter Winkel $2\theta = \SI{0}{\degree}$ beibehalten
zwischen der Probe und dem Detektor. Das Maximum aus dem resultierenden Intensitätsverlauf wird abgelesen und für die weiteren Schritte verwendet. 
Darauf folgt der zweite Z-Scan. Weil die Position der Probe sich durch den Rockingscan verändert hat, wird ein erneuter Z-Scan durchgeführt.
Hiernach wird ein zweiter Rockingscan durchgeführt für eine Erhöhung der Präzession. Es wird erneut das Maximum abgelesen während die
Motoren entsprechend ausgerichtet werden. Als letztes wird ein dritter Rockingscan durchgeführt, bei dem mit genauerer Kalibration das
Maximum vermessen wird.

\subsection{Messung}

Nach der Justage wird ein Reflektivitätsscan durchgeführt. Der Einfallswinkel $\alpha_i$ und der Winkel zwischen Probe 
und Detektor $\alpha_\text{f}$ sind hier gleich. Es wird ein Scanbereich von $\SI{0}{\degree}$ bis $\SI{2.5}{\degree}$ und eine Schrittweite 
von $\SI{0.005}{\degree}$ gewählt. Die Messzeit beträgt $\SI{40}{\minute}$. \\
Zusätzlich wird ein diffuser Scan durchgeführt. Hier wird der Anteil der gestreuten Intensität an der Reflektivität bestimmt.
