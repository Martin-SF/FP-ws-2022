\section{Diskussion}
\label{sec:Diskussion}



Zusammenfassend haben wir in unserem Praktikum gepulste NMR-Spektroskopie betrieben und die Relaxationszeiten T1 und T2 gemessen. Die erhaltenen Ergebnisse zeigen, dass T1 und T2 von verschiedenen Faktoren wie der Lage des Kerns im Molekül und seiner chemischen Umgebung abhängen. Die Messung von T1 und T2 kann daher wichtige Informationen über die Anordnung von Kernen in einem Molekül und ihre Wechselwirkungen mit der Umgebung liefern.

Im weiteren Verlauf unseres Praktikums haben wir die Relaxationszeiten T1 und T2 für verschiedene Proben untersucht und verglichen. Durch die Vergleichsanalyse konnten wir feststellen, dass T1 und T2 für verschiedene Proben unterschiedlich sind. Dies ist auf die unterschiedlichen chemischen Eigenschaften der Proben und ihre Wechselwirkungen mit der Umgebung zurückzuführen.

Um die Relaxationszeiten T1 und T2 genauer bestimmen zu können, haben wir die Temperatur der Proben variiert und die Auswirkungen auf T1 und T2 untersucht. Durch die Temperaturänderungen konnten wir beobachten, dass sich T1 und T2 verändern. Dies ist auf die thermischen Bewegungen der Kernspins und ihre Wechselwirkungen mit der Umgebung zurückzuführen.

Insgesamt haben wir in unserem Praktikum wichtige Erkenntnisse über die Relaxationszeiten T1 und T2 und ihre Abhängigkeit von verschiedenen Faktoren gewonnen. Die Messung von T1 und T2 mithilfe der NMR-Spektroskopie ist ein wertvolles Werkzeug, um die Struktur und Dynamik von Molekülen zu untersuchen.