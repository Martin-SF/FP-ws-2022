\section{Auswertung}
\label{sec:Auswertung}

Im folgenden Abschnitt werden die Ergebnisse unseres Praktikums vorgestellt und diskutiert. Wir haben gepulste NMR-Spektroskopie betrieben und die Relaxationszeiten T1 und T2 mithilfe des Inversion-Recovery-Verfahrens und des Carr-Purcell-Meiboom-Gill-Verfahrens gemessen. Die erhaltenen Ergebnisse zeigen, dass T1 und T2 von verschiedenen Faktoren abhängen, wie zum Beispiel der Lage des Kerns im Molekül, seiner chemischen Umgebung und der Temperatur.

Um die Relaxationszeiten T1 und T2 zu messen, haben wir zunächst ein Probe präpariert und in einem Kernspinresonanz-Spektrometer positioniert. Anschließend haben wir die gewünschten Magnetfeldpulse angewendet und die magnetische Ausrichtung des Kerns über die Zeit gemessen. Die erhaltenen Daten wurden dann analysiert, um T1 und T2 zu bestimmen.

Die Relaxationszeit T1 konnte mithilfe des Inversion-Recovery-Verfahrens und des Carr-Purcell-Meiboom-Gill-Verfahrens gemessen werden. Die erhaltenen Werte zeigen, dass T1 von der Lage des Kerns im Molekül und seiner chemischen Umgebung abhängt. In unseren Experimenten haben wir beobachtet, dass T1 länger ist, wenn der Kern sich in einer weniger starken chemischen Umgebung befindet.

Die Relaxationszeit T2 konnte mithilfe des Spin-Echo-Verfahrens und des Hahn-Echo-Verfahrens

gemessen werden. Die erhaltenen Werte zeigen, dass T2 von der Lage des Kerns im Molekül und seiner chemischen Umgebung abhängt. In unseren Experimenten haben wir beobachtet, dass T2 länger ist, wenn der Kern sich in einer weniger starken chemischen Umgebung befindet.
